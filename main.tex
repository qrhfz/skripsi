\documentclass[a4paper,12pt]{report}
\usepackage[T1]{fontenc}
\usepackage[utf8]{inputenc}
\usepackage{lmodern}
\usepackage{amsmath}
\usepackage{amsfonts}
\usepackage{amssymb}
\usepackage{amsthm}
\usepackage{graphicx}
\usepackage{color}
\usepackage{xcolor}
\usepackage{url}
\usepackage{textcomp}
\usepackage{parskip}
\usepackage[indonesian]{babel}
\usepackage[backend=bibtex,style=ieee]{biblatex}
\usepackage{csquotes}
\usepackage[explicit]{titlesec}
\usepackage{mathptmx}

\renewcommand{\thechapter}{\Roman{chapter}}
\renewcommand{\thesection}{\arabic{chapter}.\arabic{section}}

\titleformat{\chapter}[display]{\normalfont\Large\bfseries\centering}
{\MakeUppercase{\chaptertitlename}  \thechapter}
{0em}
{\MakeUppercase{#1}}

\titleformat{\section}
  {\normalfont\fontsize{12}{15}\bfseries}{\thesection}{0em}{ #1}

\addto\captionsindonesian{
  \renewcommand{\abstractname}{Abstraksi}
}

\titlespacing*{\chapter}{0pt}{0pt}{12pt}

\DeclareLanguageMapping{indonesian}{bahasa}

\title{Klasifikasi Berita Hoaks dengan Metode Convolutional Neural Network pada Judul Berita Berbahasa Indonesia}
\author{Qori El-Hafizh}
\date{\today}

\bibliography{text_classification}
\begin{document}

\maketitle

\begin{abstract}
\end{abstract}

\tableofcontents

\chapter{Pendahuluan}
\section{Latar Belakang Masalah}
Orang lebih banyak mendapatkan berita dari media online dibanding media tradisional seperti koran dan televisi.
Hal tersebut disebabkan karena orang lebih sering menghabiskan waktu menggunakan platform media sosial.
73\% masyarakat Indonesia mengonsumsi informasi dari media sosial.
Angka tersebut lebih tinggi dibanding jumlah masyarakat yang mengonsumsi informasi dari Televisi sebesar 59.7\%
dan media cetak sebesar 9,7\%. \cite*{katadata_73_nodate}
Kemudahan mengakses informasi melalui media online juga dibarengi dengan munculnya berita-berita bohong.
Kemenkominfo mencatat terdapat 5156 isu hoaks dari Agustus 2018 hingga Maret 2020. \cite*{kominfo_detail_nodate}

Saleh et al membuat model klasifikasi hoaks dengan Convolutinal Neural Network \cite*{saleh_opcnn-fake_2021}.
Prasetyo et al menggunakan TF-IDF dengan berbagai algoritma machine learning untuk mengklasifikasikan hoaks berbahasa Indonesia \cite*{prasetyo_evaluation_2019}.
Nayoga et al menggunakan 1 Dimension Convolutional Neural Network untuk klasifikasi hoaks \cite*{nayoga_hoax_2021}.

Pada skripsi ini saya menguji penggunaan Convolutional Neural Network untuk mengklasifikasikan hoaks.
\section{Rumusan Masalah}

Bagaimana akurasi Convolutional Neural Network untuk mengklasifikasikan judul berita sebagai hoaks atau tidak.

\section{Batasan Masalah}
Dataset berita hoax diambil dari laporan konten Hoax turnbackhoax.id dan kominfo sepanjang 2021.
Dataset berita non hoax diambil dari cnnindonesia.com.
Dataset yang digunakan terdiri dari label dan judul berita.

\section{Tujuan Penelitian}
Mengukur keakuratan CNN dalam mengklasifikasikan berita hoaks berbahasa Indonesia.
\section{Manfaat Penelitian}
Sebagai sumbangan pengetahuan bagi penyelenggara sistem elektronik dan pemerintah untuk menanggulangi hoaks.
\chapter{Landasan Teori}
\section{Tinjauan Studi}
\section{Tinjauan Pustaka}
\section{Kerangka Pemikiran}

\chapter{METODE PENELITIAN}
\section{Prosedur pengambilan data}
\section{Analisis Data}
\section{Metode}

\printbibliography

\end{document}