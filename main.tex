\documentclass[a4paper,11pt]{report}
\usepackage[T1]{fontenc}
\usepackage[utf8]{inputenc}
\usepackage{lmodern}
\usepackage{amsmath}
\usepackage{amsfonts}
\usepackage{amssymb}
\usepackage{amsthm}
\usepackage{graphicx}
\usepackage{color}
\usepackage{xcolor}
\usepackage{url}
\usepackage{textcomp}
\usepackage{parskip}
\usepackage[indonesian]{babel}
\usepackage[backend=bibtex,style=ieee]{biblatex}
\usepackage{csquotes}

\addto\captionsindonesian{%
  \renewcommand{\abstractname}{Abstrak}%
}

\DeclareLanguageMapping{indonesian}{bahasa}

\title{Topic Modeling dengan Latent Dirichlet Allocation pada Judul Berita Berbahasa Indonesia}
\author{Qori El-Hafizh}
\date{\today}

\bibliography{topic_modeling}
\begin{document}

\maketitle

\begin{abstract}
\end{abstract}

\tableofcontents

\chapter{Pendahuluan}
\section{Latar Belakang Masalah}
Topic modeling adalah \cite{surjandari_mining_2018}
\section{Rumusan Masalah}
\section{Batasan Masalah}
\section{Tujuan Penelitian}
\section{Manfaat Penelitian}

\chapter{Landasan Teori}
\section{Tinjauan Studi}
\section{Tinjauan Pustaka}
\section{Kerangka Pemikiran}

\chapter{Metode Penelitian}
\section{Prosedur pengambilan data}
\section{Analisis Data}
\section{Metode}

\printbibliography

\end{document}