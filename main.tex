\documentclass[a4paper,12pt]{report}
\usepackage[T1]{fontenc}
\usepackage[utf8]{inputenc}
\usepackage{lmodern}
\usepackage{amsmath}
\usepackage{amsfonts}
\usepackage{amssymb}
\usepackage{amsthm}
\usepackage{graphicx}
\usepackage{color}
\usepackage{xcolor}
\usepackage{url}
\usepackage{textcomp}
\usepackage{parskip}
\usepackage[indonesian]{babel}
\usepackage[backend=bibtex,style=ieee]{biblatex}
\usepackage{csquotes}
\usepackage[explicit]{titlesec}
\usepackage{mathptmx}

\renewcommand{\thechapter}{\Roman{chapter}}
\renewcommand{\thesection}{\arabic{chapter}.\arabic{section}}

\titleformat{\chapter}[display]{\normalfont\Large\bfseries\centering}
{\MakeUppercase{\chaptertitlename}  \thechapter}
{0em}
{\MakeUppercase{#1}}

\titleformat{\section}
  {\normalfont\fontsize{12}{15}\bfseries}{\thesection}{0em}{ #1}

\addto\captionsindonesian{
  \renewcommand{\abstractname}{Abstraksi}
}

\titlespacing*{\chapter}{0pt}{0pt}{12pt}

\DeclareLanguageMapping{indonesian}{bahasa}

\title{Topic Modeling dengan Latent Dirichlet Allocation pada Judul Berita Berbahasa Indonesia}
\author{Qori El-Hafizh}
\date{\today}

\bibliography{topic_modeling}
\begin{document}

\maketitle

\begin{abstract}
\end{abstract}

\tableofcontents

\chapter{Pendahuluan}
\section{Latar Belakang Masalah}
Di era digital saat ini, Pembaca media digital telah melampaui media cetak.
Menurut Kemenkominfo saat ini terdapat 100 portal berita daring yang telah terverifikasi.
Dari banyaknya portal berita online yang ada muncullah kebutuhan untuk mengetahui topik berita yang dikonsumsi masyarakat.
Hal ini dapat berguna untuk aplikasi pengumpul berita, jurnalis, hingga pembuat kebijakan.
\section{Rumusan Masalah}
\begin{enumerate}
    \item Bagaimana akurasi LDA untuk clustering judul berita?
\end{enumerate}
\section{Batasan Masalah}
\section{Tujuan Penelitian}
\section{Manfaat Penelitian}

\chapter{Landasan Teori}
\section{Tinjauan Studi}
\section{Tinjauan Pustaka}
\section{Kerangka Pemikiran}

\chapter{METODE PENELITIAN}
\section{Prosedur pengambilan data}
\section{Analisis Data}
\section{Metode}

\printbibliography

\end{document}